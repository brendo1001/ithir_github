\nonstopmode{}
\documentclass[a4paper]{book}
\usepackage[times,inconsolata,hyper]{Rd}
\usepackage{makeidx}
\usepackage[utf8]{inputenc} % @SET ENCODING@
% \usepackage{graphicx} % @USE GRAPHICX@
\makeindex{}
\begin{document}
\chapter*{}
\begin{center}
{\textbf{\huge Package `ithir'}}
\par\bigskip{\large \today}
\end{center}
\inputencoding{utf8}
\ifthenelse{\boolean{Rd@use@hyper}}{\hypersetup{pdftitle = {ithir: Tools and Data for Digital Soil Informatics}}}{}
\begin{description}
\raggedright{}
\item[Type]\AsIs{Package}
\item[Title]\AsIs{Tools and Data for Digital Soil Informatics}
\item[Version]\AsIs{1.0.0}
\item[Date]\AsIs{2025-05-20}
\item[Author]\AsIs{Brendan Malone [aut, cre]}
\item[Maintainer]\AsIs{Brendan Malone }\email{brendan.malone@csiro.au}\AsIs{}
\item[Description]\AsIs{Provides tools and datasets to support digital soil mapping and analysis workflows. Includes fast mass-preserving spline functions for raster and profile data, model evaluation metrics for both continuous and categorical predictions, and example datasets for demonstration and testing.}
\item[Depends]\AsIs{R (>= 4.0)}
\item[Imports]\AsIs{methods, terra, MASS, aqp}
\item[Suggests]\AsIs{sf, graphics, stats, utils, testthat}
\item[License]\AsIs{GPL-2}
\item[Encoding]\AsIs{UTF-8}
\item[LazyData]\AsIs{true}
\item[LazyDataCompression]\AsIs{xz}
\item[RoxygenNote]\AsIs{7.2.3}
\item[NeedsCompilation]\AsIs{no}
\end{description}
\Rdcontents{\R{} topics documented:}
\inputencoding{utf8}
\HeaderA{bbRaster}{Get Bounding Box Coordinates from a \code{SpatRaster}}{bbRaster}
\keyword{methods}{bbRaster}
%
\begin{Description}
Returns the bounding box of a \code{SpatRaster} object as a 4 x 2 matrix, with each row representing one corner of the extent in coordinate pairs.
\end{Description}
%
\begin{Usage}
\begin{verbatim}
bbRaster(obj)
\end{verbatim}
\end{Usage}
%
\begin{Arguments}
\begin{ldescription}
\item[\code{obj}] An object of class \code{SpatRaster}, from the \pkg{terra} package.
\end{ldescription}
\end{Arguments}
%
\begin{Value}
A 4 x 2 numeric matrix. Each row corresponds to a corner of the bounding box, in clockwise order starting from the lower left:
\begin{description}

\item[Row 1] Lower-left (xmin, ymin)
\item[Row 2] Lower-right (xmax, ymin)
\item[Row 3] Upper-right (xmax, ymax)
\item[Row 4] Upper-left (xmin, ymax)

\end{description}

\end{Value}
%
\begin{Author}
Brendan Malone
\end{Author}
%
\begin{Examples}
\begin{ExampleCode}
library(terra)

# Example raster
target <- rast(system.file("extdata/edgeTarget_C.tif", package = "ithir"))

# Get bounding box coordinates
bbRaster(target)
\end{ExampleCode}
\end{Examples}
\inputencoding{utf8}
\HeaderA{ea\_rasSp\_fast}{Fast Mass-Preserving Spline on Raster Soil Data}{ea.Rul.rasSp.Rul.fast}
\keyword{methods}{ea\_rasSp\_fast}
%
\begin{Description}
Fits a mass-preserving spline model to a multi-layer soil property \code{SpatRaster} and returns raster layers containing spline-averaged values over user-defined depth intervals.
This function uses precomputed spline matrices for efficiency and applies the spline cell-wise using \code{terra::app()}.
\end{Description}
%
\begin{Usage}
\begin{verbatim}
ea_rasSp_fast(obj, lam = 0.1, dIn, dOut, vlow = 0, vhigh = 100, depth_res = 1)
\end{verbatim}
\end{Usage}
%
\begin{Arguments}
\begin{ldescription}
\item[\code{obj}] object of class \code{"SpatRaster"} with one layer per input depth interval.
\item[\code{lam}] spline stiffness parameter (\eqn{\lambda}{}); smaller values = smoother splines.
\item[\code{dIn}] numeric vector of input depth boundaries (e.g. \code{c(0,5,15,...)}).
\item[\code{dOut}] numeric vector defining output depth intervals (e.g. \code{c(0,30,60)}).
\item[\code{vlow}] minimum bound to truncate spline predictions (default = 0).
\item[\code{vhigh}] maximum bound to truncate spline predictions (default = 100).
\item[\code{depth\_res}] numeric; depth interpolation resolution (e.g. 1 = 1cm, 5 = 5cm).
\end{ldescription}
\end{Arguments}
%
\begin{Value}
Returns a \code{SpatRaster} with one layer per output depth interval.
\end{Value}
%
\begin{Author}
Brendan Malone
\end{Author}
%
\begin{Examples}
\begin{ExampleCode}
# Not run on CRAN due to file size and download time
## Not run: 
library(terra)

# Define SLGA V2 clay layer URLs (0–200 cm depth range)
clay_urls <- c(
  '/vsicurl/https://esoil.io/TERNLandscapes/Public/Products/TERN/SLGA/CLY/CLY_000_005_EV_N_P_AU_TRN_N_20210902.tif',
  '/vsicurl/https://esoil.io/TERNLandscapes/Public/Products/TERN/SLGA/CLY/CLY_005_015_EV_N_P_AU_TRN_N_20210902.tif',
  '/vsicurl/https://esoil.io/TERNLandscapes/Public/Products/TERN/SLGA/CLY/CLY_015_030_EV_N_P_AU_TRN_N_20210902.tif',
  '/vsicurl/https://esoil.io/TERNLandscapes/Public/Products/TERN/SLGA/CLY/CLY_030_060_EV_N_P_AU_TRN_N_20210902.tif',
  '/vsicurl/https://esoil.io/TERNLandscapes/Public/Products/TERN/SLGA/CLY/CLY_060_100_EV_N_P_AU_TRN_N_20210902.tif',
  '/vsicurl/https://esoil.io/TERNLandscapes/Public/Products/TERN/SLGA/CLY/CLY_100_200_EV_N_P_AU_TRN_N_20210902.tif'
)

# Load and crop a small extent near Canberra
clay_stack <- rast(clay_urls)
aoi <- ext(149.00, 149.10, -36.00, -35.90)
clay_crop <- crop(clay_stack, aoi)

# Fit spline and generate interpolated output
out <- ea_rasSp_fast(
  obj = clay_crop,
  lam = 0.1,
  dIn = c(0, 5, 15, 30, 60, 100, 200),
  dOut = c(0, 30, 60),
  depth_res = 2
)

# Plot the result
plot(out)

## End(Not run)
\end{ExampleCode}
\end{Examples}
\inputencoding{utf8}
\HeaderA{ea\_spline}{Fit a Mass-Preserving Spline to Soil Profile Data}{ea.Rul.spline}
\keyword{methods}{ea\_spline}
%
\begin{Description}
Fits a continuous mass-preserving spline to numeric soil profile data, typically horizon-structured, such as soil organic carbon or pH. The spline ensures that the integral (or mass) of the fitted curve over each horizon matches the observed values, enabling harmonisation to standard depths.
\end{Description}
%
\begin{Usage}
\begin{verbatim}
ea_spline(obj, var.name, lam = 0.1, d = c(0, 5, 15, 30, 60, 100, 200),
          vlow = 0, vhigh = 1000, show.progress = TRUE)
\end{verbatim}
\end{Usage}
%
\begin{Arguments}
\begin{ldescription}
\item[\code{obj}] An object of class \code{data.frame} or \code{SoilProfileCollection}. Must contain top and bottom depth values and a target numeric variable.
\item[\code{var.name}] Character string; name of the target numeric variable to fit the spline on.
\item[\code{lam}] Numeric; smoothing parameter (lambda). Lower values give smoother splines.
\item[\code{d}] Numeric vector of standard depths at which to estimate the harmonised values.
\item[\code{vlow}] Numeric; lower bound. Values below this will be set to \code{vlow}.
\item[\code{vhigh}] Numeric; upper bound. Values above this will be set to \code{vhigh}.
\item[\code{show.progress}] Logical; if \code{TRUE}, displays a progress bar.
\end{ldescription}
\end{Arguments}
%
\begin{Value}
A list with the following components:
\begin{description}

\item[\code{harmonised}] A data frame of spline-estimated values at the specified standard depths.
\item[\code{obs.preds}] A data frame of observed values and corresponding spline predictions by profile and depth.
\item[\code{var.1cm}] A matrix of fitted values at 1 cm increments, for each profile.
\item[\code{splineFitError}] A data frame with root mean square error (RMSE) and RMSE divided by IQR for each profile.

\end{description}

\end{Value}
%
\begin{Note}
The function requires at least two horizons with valid numeric values for spline fitting. Profiles with only one horizon are handled using value replication (no spline fitting). Only positive values for top and bottom depths are accepted. Horizon values are assumed to be block-supported averages.
\end{Note}
%
\begin{Author}
Brendan Malone
\end{Author}
%
\begin{References}
\begin{itemize}

\item{} Bishop, T.F.A., McBratney, A.B., Laslett, G.M. (1999). \Rhref{http://dx.doi.org/10.1016/S0016-7061(99)00003-8}{Modelling soil attribute depth functions with equal-area quadratic smoothing splines}. Geoderma, 91(1–2), 27–45.
\item{} Malone, B.P., McBratney, A.B., Minasny, B., Laslett, G.M. (2009). \Rhref{http://dx.doi.org/10.1016/j.geoderma.2009.10.007}{Mapping continuous depth functions of soil carbon storage and available water capacity}. Geoderma, 154(1–2), 138–152.

\end{itemize}

\end{References}
%
\begin{Examples}
\begin{ExampleCode}
# Example using a simple data.frame
data(oneProfile)
str(oneProfile)
sp_fit <- ea_spline(obj = oneProfile, var.name = "C.kg.m3.")

# Example using a SoilProfileCollection from the aqp package
# library(aqp)
# library(plyr)
# lon <- 3.90; lat <- 7.50; id <- "ISRIC:NG0017"
# top <- c(0, 18, 36, 65, 87, 127)
# bottom <- c(18, 36, 65, 87, 127, 181)
# ORCDRC <- c(18.4, 4.4, 3.6, 3.6, 3.2, 1.2)
# munsell <- c("7.5YR3/2", "7.5YR4/4", "2.5YR5/6", "5YR5/8", "5YR5/4", "10YR7/3")
# prof1 <- join(data.frame(id, top, bottom, ORCDRC, munsell),
#               data.frame(id, lon, lat), type = 'inner')
# depths(prof1) <- id ~ top + bottom
# site(prof1) <- ~ lon + lat
# ORCDRC.s <- ea_spline(prof1, var.name = "ORCDRC")
# str(ORCDRC.s)
\end{ExampleCode}
\end{Examples}
\inputencoding{utf8}
\HeaderA{edgeLandClass}{Land Classification Points from the Edgeroi District, NSW}{edgeLandClass}
\keyword{datasets}{edgeLandClass}
%
\begin{Description}
A \code{data.frame} of 500 point locations in the Edgeroi District, NSW, Australia (approx. 30.11°S, 149.66°E), each assigned to one of six estimated land classes. The land classification was derived via unsupervised classification of Landsat 7 ETM+ spectral data (acquisition date unknown). These points represent a random sample from a classified land use map.
\end{Description}
%
\begin{Usage}
\begin{verbatim}
data(edgeLandClass)
\end{verbatim}
\end{Usage}
%
\begin{Format}
A \code{data.frame} with 500 rows and 3 columns:
\begin{description}

\item[\code{x}] Easting (UTM Zone 55)
\item[\code{y}] Northing (UTM Zone 55)
\item[\code{LandClass}] Integer values from 1 to 6 representing land classes: 
\begin{enumerate}

\item{} Dense forest
\item{} Open forest
\item{} Water bodies
\item{} Woody vegetation and native grassland
\item{} Irrigated cropping
\item{} Dryland cropping

\end{enumerate}



\end{description}

\end{Format}
%
\begin{Details}
This dataset represents typical point-based environmental classification data and is suitable for supervised modelling, classification accuracy assessment, or land use interpretation.
\end{Details}
%
\begin{References}
\begin{itemize}

\item{} Soil Security Laboratory, 2015. \emph{Use R for Digital Soil Mapping Manual}. The University of Sydney, Sydney, Australia.

\end{itemize}

\end{References}
%
\begin{Examples}
\begin{ExampleCode}
data(edgeLandClass)

# View land class summary
summary(edgeLandClass$LandClass)

# Plot the locations by class
plot(edgeLandClass$x, edgeLandClass$y, col = edgeLandClass$LandClass,
     pch = 20, main = "Edgeroi Land Classification Points")
\end{ExampleCode}
\end{Examples}
\inputencoding{utf8}
\HeaderA{edgeroiCovariates}{Environmental Covariate Rasters for the Full Edgeroi District, NSW}{edgeroiCovariates}
\aliasA{edgeroiCovariates\_elevation.tif}{edgeroiCovariates}{edgeroiCovariates.Rul.elevation.tif}
\aliasA{edgeroiCovariates\_landsat\_b3.tif}{edgeroiCovariates}{edgeroiCovariates.Rul.landsat.Rul.b3.tif}
\aliasA{edgeroiCovariates\_landsat\_b4.tif}{edgeroiCovariates}{edgeroiCovariates.Rul.landsat.Rul.b4.tif}
\aliasA{edgeroiCovariates\_radK.tif}{edgeroiCovariates}{edgeroiCovariates.Rul.radK.tif}
\aliasA{edgeroiCovariates\_twi.tif}{edgeroiCovariates}{edgeroiCovariates.Rul.twi.tif}
\keyword{datasets}{edgeroiCovariates}
%
\begin{Description}
GeoTIFF raster files of selected environmental covariates covering the full Edgeroi District in northern New South Wales, Australia. These spatial layers are commonly used in digital soil mapping as predictor variables.
\end{Description}
%
\begin{Format}
This dataset consists of five raster layers (GeoTIFF format) with a pixel resolution of 90 m. The following files are included:
\begin{description}

\item[\code{edgeroiCovariates\_elevation.tif}] Ground elevation (bare earth DEM)
\item[\code{edgeroiCovariates\_twi.tif}] Topographic wetness index derived from the DEM
\item[\code{edgeroiCovariates\_radK.tif}] Gamma radiometric potassium (K) dose rate
\item[\code{edgeroiCovariates\_landsat\_b3.tif}] Band 3 reflectance from Landsat 7 ETM+
\item[\code{edgeroiCovariates\_landsat\_b4.tif}] Band 4 reflectance from Landsat 7 ETM+

\end{description}

\end{Format}
%
\begin{Details}
The Edgeroi District is a well-studied alluvial cropping area situated on the Namoi River plain. It has been the focus of numerous soil investigations, including systematic sampling on a triangular grid (McGarry et al., 1989) and subsequent digital soil mapping applications (Malone et al., 2009). These rasters provide spatial covariates that are commonly used for soil attribute modelling.
\end{Details}
%
\begin{Note}
Raster files are stored in the \code{inst/extdata} directory of the package and can be accessed with \code{system.file()}. For example:

\code{system.file("extdata/edgeroiCovariates\_elevation.tif", package = "ithir")}

All rasters use WGS84 / UTM Zone 55.
\end{Note}
%
\begin{References}
\begin{itemize}

\item{} Malone, B.P., McBratney, A.B., Minasny, B. (2009). \Rhref{http://dx.doi.org/10.1016/j.geoderma.2009.10.007}{Mapping continuous depth functions of soil carbon storage and available water capacity}. Geoderma, 154, 138–152.
\item{} McGarry, D., Ward, W.T., McBratney, A.B. (1989). \emph{Soil Studies in the Lower Namoi Valley: Methods and Data. The Edgeroi Data Set}. (2 vols). CSIRO Division of Soils: Adelaide.

\end{itemize}

\end{References}
%
\begin{Examples}
\begin{ExampleCode}
library(terra)

# Load elevation layer
elev_path <- system.file("extdata/edgeroiCovariates_elevation.tif", package = "ithir")
elev_rast <- rast(elev_path)
plot(elev_rast, main = "Edgeroi Elevation Map")
\end{ExampleCode}
\end{Examples}
\inputencoding{utf8}
\HeaderA{edgeroi\_covariates\_subset}{Selected Subset of Environmental Covariates for the Edgeroi District, NSW}{edgeroi.Rul.covariates.Rul.subset}
\aliasA{edgeGrids\_Doserate}{edgeroi\_covariates\_subset}{edgeGrids.Rul.Doserate}
\aliasA{edgeGrids\_Elevation}{edgeroi\_covariates\_subset}{edgeGrids.Rul.Elevation}
\aliasA{edgeGrids\_Panchromat}{edgeroi\_covariates\_subset}{edgeGrids.Rul.Panchromat}
\aliasA{edgeGrids\_Slope}{edgeroi\_covariates\_subset}{edgeGrids.Rul.Slope}
\aliasA{edgeGrids\_TWI}{edgeroi\_covariates\_subset}{edgeGrids.Rul.TWI}
\keyword{datasets}{edgeroi\_covariates\_subset}
%
\begin{Description}
A set of GeoTIFF rasters representing selected environmental covariates for a small area in the Edgeroi District, New South Wales, Australia.
\end{Description}
%
\begin{Format}
The rasters are stored as GeoTIFF files with a spatial resolution of 90 m × 90 m. The projection is WGS 84 UTM Zone 55. The following rasters are available:
\begin{itemize}

\item{} \code{edgeGrids\_Doserate.tif} — Gamma radiometric data.
\item{} \code{edgeGrids\_Elevation.tif} — Ground elevation derived from a digital elevation model (DEM).
\item{} \code{edgeGrids\_Panchromat.tif} — Panchromatic band from Landsat 7 ETM+ imagery.
\item{} \code{edgeGrids\_Slope.tif} — Slope gradient derived from the DEM.
\item{} \code{edgeGrids\_TWI.tif} — Topographic wetness index (TWI), derived from the DEM.

\end{itemize}

\end{Format}
%
\begin{Details}
The Edgeroi District is located on the alluvial Namoi River plain and is a highly productive agricultural region. It has been the focus of numerous soil investigations, including those described in McGarry et al. (1989). This dataset supports digital soil mapping (DSM) applications and is a spatial subset of a more extensive covariate dataset for the district.
\end{Details}
%
\begin{Note}
The original spatial data used to generate these rasters were sourced from public repositories maintained by CSIRO, Geoscience Australia, and NASA. All rasters use the WGS 84 UTM Zone 55 coordinate reference system.
\end{Note}
%
\begin{References}
\begin{itemize}

\item{} Malone, B.P., McBratney, A.B., Minasny, B. (2009). \Rhref{http://dx.doi.org/10.1016/j.geoderma.2009.10.007}{Mapping continuous depth functions of soil carbon storage and available water capacity}. Geoderma, 154, 138–152.
\item{} McGarry, D., Ward, W.T., McBratney, A.B. (1989). Soil Studies in the Lower Namoi Valley: Methods and Data. The Edgeroi Data Set. CSIRO Division of Soils.

\end{itemize}

\end{References}
%
\begin{Examples}
\begin{ExampleCode}
library(ithir)
library(terra)

# Load and plot the elevation raster
elevation <- rast(system.file("extdata/edgeGrids_Elevation.tif", package = "ithir"))
plot(elevation, main = "Edgeroi Elevation Map")
\end{ExampleCode}
\end{Examples}
\inputencoding{utf8}
\HeaderA{edgeroi\_splineCarbon}{Harmonised Soil Carbon Density Data from the Edgeroi District, NSW}{edgeroi.Rul.splineCarbon}
\keyword{datasets}{edgeroi\_splineCarbon}
%
\begin{Description}
A \code{data.frame} containing soil carbon density estimates at 341 locations across the Edgeroi District, NSW, Australia (approx. 30.11°S, 149.66°E). Soil carbon values were derived using a pedotransfer function based on measured soil carbon concentration and texture. The resulting values were harmonised using mass-preserving splines according to the GlobalSoilMap specifications.
\end{Description}
%
\begin{Usage}
\begin{verbatim}
data(edgeroi_splineCarbon)
\end{verbatim}
\end{Usage}
%
\begin{Format}
A \code{data.frame} with 341 rows. Columns include:
\begin{description}

\item[\code{id}] Profile identifier
\item[\code{east}] Easting (UTM Zone 55)
\item[\code{north}] Northing (UTM Zone 55)
\item[\code{carbon\_0\_5cm}] Estimated carbon density (0–5 cm)
\item[\code{carbon\_5\_15cm}] Estimated carbon density (5–15 cm)
\item[\code{carbon\_15\_30cm}] Estimated carbon density (15–30 cm)
\item[\code{carbon\_30\_60cm}] Estimated carbon density (30–60 cm)
\item[\code{carbon\_60\_100cm}] Estimated carbon density (60–100 cm)
\item[\code{carbon\_100\_200cm}] Estimated carbon density (100–200 cm)
\item[\code{max\_depth}] Maximum spline depth fitted for each profile (cm)

\end{description}

\end{Format}
%
\begin{Details}
Sampling was based on a design described by McGarry et al. (1989), with 210 samples taken from a systematic equilateral triangular grid (2.8 km spacing), and 131 additional samples located along transects or more irregularly. Spline harmonisation was applied to align depths with GlobalSoilMap conventions.
\end{Details}
%
\begin{References}
\begin{itemize}

\item{} Arrouays, D., McKenzie, N., Hempel, J., Richer de Forges, A., and McBratney, A. (eds) (2014). \emph{GlobalSoilMap: Basis of the Global Spatial Soil Information System}. CRC Press.
\item{} McGarry, D., Ward, W.T., McBratney, A.B. (1989). \emph{Soil Studies in the Lower Namoi Valley: Methods and Data. The Edgeroi Data Set}. (2 vols). CSIRO Division of Soils, Adelaide.

\end{itemize}

\end{References}
%
\begin{Examples}
\begin{ExampleCode}
library(ithir)
library(sf)

# Load the dataset
data(edgeroi_splineCarbon)

# Convert to sf object and plot
spat_edgeroi <- st_as_sf(edgeroi_splineCarbon, coords = c("east", "north"), crs = 32755)
plot(spat_edgeroi, pch = 16, cex = 0.5, main = "Edgeroi Soil Carbon Point Locations")
\end{ExampleCode}
\end{Examples}
\inputencoding{utf8}
\HeaderA{edgeTarget\_C}{1 km Resolution Soil Carbon Stock Map (Subset) – Edgeroi District, NSW}{edgeTarget.Rul.C}
\methaliasA{edgeTarget\_C.tif}{edgeTarget\_C}{edgeTarget.Rul.C.tif}
\keyword{datasets}{edgeTarget\_C}
%
\begin{Description}
A GeoTIFF raster representing predicted soil organic carbon (SOC) stock for the 0–30 cm depth interval over a small subset of the Edgeroi District, NSW, Australia. The map is at 1 km spatial resolution and was derived using digital soil mapping methods.
\end{Description}
%
\begin{Format}
A single-layer raster (GeoTIFF) with 7 rows and 10 columns, each cell representing soil carbon stock in the 0–30 cm layer (e.g., Mg C/ha). The spatial resolution is 1 km x 1 km. The raster is projected in WGS84 / UTM Zone 55.
\end{Format}
%
\begin{Details}
This raster was produced using digital soil mapping techniques described in McBratney et al. (2003), and uses soil profile data from McGarry et al. (1989) combined with environmental covariates. The extent of the raster matches that of the covariate subset rasters provided in this package (see \code{edgeGrids\_*}), though the resolution is coarser (1 km).
\end{Details}
%
\begin{Note}
The raster file is located in the \code{inst/extdata} directory and can be accessed using \code{system.file()}. It shares spatial coverage with the covariates in \code{edgeroi\_covariates\_subset} but has a coarser resolution.
\end{Note}
%
\begin{References}
\begin{itemize}

\item{} Malone, B.P., McBratney, A.B., Minasny, B. (2009). \Rhref{http://dx.doi.org/10.1016/j.geoderma.2009.10.007}{Mapping continuous depth functions of soil carbon storage and available water capacity}. Geoderma, 154, 138–152.
\item{} McBratney, A.B., Mendonça Santos, M.L., Minasny, B. (2003). \Rhref{http://dx.doi.org/10.1016/S0016-7061(03)00223-4}{On digital soil mapping}. Geoderma, 117, 3–52.
\item{} McGarry, D., Ward, W.T., McBratney, A.B. (1989). \emph{Soil Studies in the Lower Namoi Valley: Methods and Data. The Edgeroi Data Set}. (2 vols). CSIRO Division of Soils: Adelaide.

\end{itemize}

\end{References}
%
\begin{Examples}
\begin{ExampleCode}
library(terra)

# Load the raster from the package
soc_path <- system.file("extdata/edgeTarget_C.tif", package = "ithir")
soc_raster <- rast(soc_path)

# Plot the raster
plot(soc_raster, main = "Edgeroi SOC Stock (0–30 cm)", col = rev(terrain.colors(20)))
\end{ExampleCode}
\end{Examples}
\inputencoding{utf8}
\HeaderA{fit\_mpspline\_optimized}{Fit Mass-Preserving Spline to a Single Soil Profile}{fit.Rul.mpspline.Rul.optimized}
\keyword{methods}{fit\_mpspline\_optimized}
%
\begin{Description}
Applies the mass-preserving spline algorithm to a numeric vector of soil profile values and returns averaged values over specified output depth intervals. 
It uses precomputed matrix structures for efficiency.
\end{Description}
%
\begin{Usage}
\begin{verbatim}
fit_mpspline_optimized(vals, spline_info, dOut, vlow, vhigh, depth_res = 1)
\end{verbatim}
\end{Usage}
%
\begin{Arguments}
\begin{ldescription}
\item[\code{vals}] numeric vector of input values across input soil depth intervals (e.g., from one raster pixel).
\item[\code{spline\_info}] list output from \code{\LinkA{precompute\_spline\_structures}{precompute.Rul.spline.Rul.structures}} that contains spline matrix components.
\item[\code{dOut}] numeric vector defining output depth intervals (e.g., \code{c(0,30,60)}).
\item[\code{vlow}] minimum bound to truncate spline predictions (e.g., 0).
\item[\code{vhigh}] maximum bound to truncate spline predictions (e.g., 100).
\item[\code{depth\_res}] numeric; resolution for interpolating spline (e.g., 1 = 1cm steps).
\end{ldescription}
\end{Arguments}
%
\begin{Value}
Returns a numeric vector of spline-averaged values for each specified output depth interval.
\end{Value}
%
\begin{Author}
Brendan Malone
\end{Author}
%
\begin{SeeAlso}
\code{\LinkA{precompute\_spline\_structures}{precompute.Rul.spline.Rul.structures}}, \code{\LinkA{ea\_rasSp\_fast}{ea.Rul.rasSp.Rul.fast}}
\end{SeeAlso}
%
\begin{Examples}
\begin{ExampleCode}
# Not run on CRAN due to external raster data size
## Not run: 
library(terra)

# Define SLGA clay raster URLs
clay_urls <- c(
  '/vsicurl/https://esoil.io/TERNLandscapes/Public/Products/TERN/SLGA/CLY/CLY_000_005_EV_N_P_AU_TRN_N_20210902.tif',
  '/vsicurl/https://esoil.io/TERNLandscapes/Public/Products/TERN/SLGA/CLY/CLY_005_015_EV_N_P_AU_TRN_N_20210902.tif',
  '/vsicurl/https://esoil.io/TERNLandscapes/Public/Products/TERN/SLGA/CLY/CLY_015_030_EV_N_P_AU_TRN_N_20210902.tif',
  '/vsicurl/https://esoil.io/TERNLandscapes/Public/Products/TERN/SLGA/CLY/CLY_030_060_EV_N_P_AU_TRN_N_20210902.tif',
  '/vsicurl/https://esoil.io/TERNLandscapes/Public/Products/TERN/SLGA/CLY/CLY_060_100_EV_N_P_AU_TRN_N_20210902.tif',
  '/vsicurl/https://esoil.io/TERNLandscapes/Public/Products/TERN/SLGA/CLY/CLY_100_200_EV_N_P_AU_TRN_N_20210902.tif'
)

# Load and crop raster stack
clay_stack <- rast(clay_urls)
aoi <- ext(149.00, 149.10, -36.00, -35.90)
clay_crop <- crop(clay_stack, aoi)

# Extract a single profile (pixel)
vals <- terra::extract(clay_crop, cbind(149.05, -35.95))[1, -1]

# Precompute spline structures
dIn <- c(0, 5, 15, 30, 60, 100, 200)
spline_info <- precompute_spline_structures(dIn, lam = 0.1)

# Fit spline to single profile
fit <- fit_mpspline_optimized(
  vals = vals,
  spline_info = spline_info,
  dOut = c(0, 30, 60),
  vlow = 0,
  vhigh = 100,
  depth_res = 1
)

fit

## End(Not run)
\end{ExampleCode}
\end{Examples}
\inputencoding{utf8}
\HeaderA{fuzzyEx}{Derivation of Fuzzy Memberships to Classes with Extragrades}{fuzzyEx}
\keyword{methods}{fuzzyEx}
%
\begin{Description}
This function computes fuzzy membership values for a set of multivariate observations given a set of class centroids, based on the fuzzy k-means with extragrades algorithm (McBratney and de Gruijter, 1992). It is designed to complement outputs from the FuzMe software (Minasny and McBratney, 2002).
\end{Description}
%
\begin{Usage}
\begin{verbatim}
fuzzyEx(data, centroid, cv, expon, alfa)
\end{verbatim}
\end{Usage}
%
\begin{Arguments}
\begin{ldescription}
\item[\code{data}] A \code{data.frame} where the first column is an observation identifier, and the remaining columns contain numeric values for the variables used in the classification.
\item[\code{centroid}] A numeric matrix of class centroids (excluding extragrade).
\item[\code{cv}] A variance-covariance matrix used to compute Mahalanobis distances between observations and class centroids.
\item[\code{expon}] A numeric value specifying the fuzzy exponent (e.g., 1.2 or 2).
\item[\code{alfa}] A numeric scalar indicating the membership threshold for the extragrade class, as returned from FuzMe.
\end{ldescription}
\end{Arguments}
%
\begin{Value}
A numeric matrix with one row per observation and \code{n + 2} columns, where \code{n} is the number of classes. The columns include:
\begin{description}

\item[Class 1 to Class n] Estimated fuzzy membership to each class
\item[Extragrade] Estimated membership to the extragrade class
\item[MaxClass] The class (including extragrade) with the highest membership for each observation

\end{description}

\end{Value}
%
\begin{Note}
Mahalanobis distance is used to evaluate dissimilarity between observations and centroids, incorporating variable covariance structure.
\end{Note}
%
\begin{Author}
Brendan Malone
\end{Author}
%
\begin{References}
\begin{itemize}

\item{} McBratney, A.B., de Gruijter, J.J. (1992). \Rhref{http://dx.doi.org/10.1111/j.1365-2389.1992.tb00127.x}{Continuum approach to soil classification by modified fuzzy k-means with extragrades}. Journal of Soil Science, 43, 159–175.
\item{} Minasny, B., McBratney, A.B. (2002). \Rhref{http://sydney.edu.au/agriculture/pal/software/fuzme.shtml}{FuzMe version 3.0}. Australian Centre for Precision Agriculture, The University of Sydney.

\end{itemize}

\end{References}
%
\begin{Examples}
\begin{ExampleCode}
## Not run: 
# Example (requires compatible centroid and covariance data)
# memberships <- fuzzyEx(data = input_data, 
#                        centroid = centroid_matrix, 
#                        cv = cov_matrix, 
#                        expon = 2, 
#                        alfa = 0.01)
## End(Not run)
\end{ExampleCode}
\end{Examples}
\inputencoding{utf8}
\HeaderA{goof}{Goodness of Fit Measures}{goof}
\keyword{methods}{goof}
%
\begin{Description}
Computes a suite of goodness-of-fit statistics for model evaluation, including root mean square error (RMSE), mean square error (MSE), prediction bias, coefficient of determination (R-squared), concordance correlation coefficient (CCC), ratio of performance to deviation (RPD), ratio of performance to interquartile distance (RPIQ), and residual variance estimates. These metrics are widely used in digital soil mapping and chemometric modelling to assess model accuracy and reliability.
\end{Description}
%
\begin{Usage}
\begin{verbatim}
goof(observed, predicted, plot.it = FALSE, type = "DSM")
\end{verbatim}
\end{Usage}
%
\begin{Arguments}
\begin{ldescription}
\item[\code{observed}] Numeric; a vector or matrix of observed values.
\item[\code{predicted}] Numeric; a vector or matrix of predicted values.
\item[\code{plot.it}] Logical; if \code{TRUE}, an xy-plot of observed vs predicted values is generated.
\item[\code{type}] Character; choose either \code{"DSM"} (default) or \code{"spec"} to tailor the output to digital soil mapping or soil spectroscopy use cases.
\end{ldescription}
\end{Arguments}
%
\begin{Value}
A \code{data.frame} containing the following goodness-of-fit statistics:
\begin{description}

\item[\code{R2}] Coefficient of determination
\item[\code{concordance}] Lin’s concordance correlation coefficient
\item[\code{MSE}] Mean square error
\item[\code{RMSE}] Root mean square error
\item[\code{bias}] Prediction bias (mean error)
\item[\code{MSEc}] Mean residual variance
\item[\code{RMSEc}] Root mean residual variance
\item[\code{RPD}] Ratio of performance to standard deviation
\item[\code{RPIQ}] Ratio of performance to interquartile range

\end{description}

If \code{plot.it = TRUE}, a scatter plot of predicted vs observed values is also returned.
\end{Value}
%
\begin{Note}
These goodness-of-fit statistics are commonly used across various modelling domains, including but not limited to digital soil mapping and chemometric spectroscopy.
\end{Note}
%
\begin{Author}
Brendan Malone
\end{Author}
%
\begin{References}
\begin{itemize}

\item{} Bellon-Maurel, V., Fernandez-Ahumada, E., Palagos, B., Roger, J., McBratney, A.B. (2010). \Rhref{http://dx.doi.org/10.1016/j.trac.2010.05.006}{Critical review of chemometric indicators commonly used for assessing the quality of the prediction of soil attributes by NIR spectroscopy}. \emph{Trends in Analytical Chemistry}, 29(9), 1073–1081.
\item{} Hastie, T., Tibshirani, R., Friedman, J. (2009). \emph{The Elements of Statistical Learning}. Springer Series in Statistics.

\end{itemize}

\end{References}
%
\begin{Examples}
\begin{ExampleCode}
library(ithir)
library(MASS)

# Load sample soil data
data(USYD_soil1)

# Fit a linear model
mod.1 <- lm(CEC ~ clay, data = USYD_soil1, y = TRUE, x = TRUE)

# Calculate goodness-of-fit statistics and plot
goof(observed = mod.1$y, predicted = mod.1$fitted.values, plot.it = TRUE)
\end{ExampleCode}
\end{Examples}
\inputencoding{utf8}
\HeaderA{goofcat}{Goodness of Fit Measures for Categorical Models}{goofcat}
\keyword{methods}{goofcat}
%
\begin{Description}
This function computes several diagnostic statistics for evaluating the performance of classification models where the target variable is categorical. The metrics include: Overall Accuracy, Producer’s Accuracy, User’s Accuracy, and the Kappa Statistic. These are widely used in remote sensing and classification modelling (see Congalton, 1991).
\end{Description}
%
\begin{Usage}
\begin{verbatim}
goofcat(observed = NULL, predicted = NULL, conf.mat, imp = FALSE)
\end{verbatim}
\end{Usage}
%
\begin{Arguments}
\begin{ldescription}
\item[\code{observed}] A vector of observed class labels (numeric or character).
\item[\code{predicted}] A vector of predicted class labels (numeric or character). Must be the same length as \code{observed}.
\item[\code{conf.mat}] Optional; a confusion matrix summarizing classification outcomes. Should be a square matrix.
\item[\code{imp}] Logical; set to \code{TRUE} if providing a pre-computed confusion matrix via \code{conf.mat}. Default is \code{FALSE}.
\end{ldescription}
\end{Arguments}
%
\begin{Value}
A named \code{list} containing:
\begin{description}

\item[\code{OverallAccuracy}] Proportion of correct classifications
\item[\code{ProducersAccuracy}] Recall or sensitivity per class
\item[\code{UsersAccuracy}] Precision or positive predictive value per class
\item[\code{Kappa}] Kappa statistic (chance-corrected agreement)
\item[\code{ConfusionMatrix}] Returned confusion matrix used in the calculation

\end{description}

\end{Value}
%
\begin{Note}
If \code{conf.mat} is not of class \code{matrix} or is not square in dimension, the function will halt with an error.
\end{Note}
%
\begin{Author}
Brendan Malone
\end{Author}
%
\begin{References}
\begin{itemize}

\item{} Congalton, R. G. (1991). \Rhref{http://dx.doi.org/10.1016/0034-4257(91)90048-B}{A review of assessing the accuracy of classifications of remotely sensed data}. \emph{Remote Sensing of Environment}, 37, 35–46.

\end{itemize}

\end{References}
%
\begin{Examples}
\begin{ExampleCode}
library(ithir)

# Using a pre-constructed confusion matrix
con.mat <- matrix(c(5, 0, 1, 2,
                    0, 15, 0, 5,
                    0, 1, 31, 0,
                    0, 10, 2, 11), nrow = 4, byrow = TRUE)
rownames(con.mat) <- colnames(con.mat) <- c("DE", "VE", "CH", "KU")
goofcat(conf.mat = con.mat, imp = TRUE)

# Using vectors of observed and predicted values
set.seed(123)
observed <- sample(1:5, 1000, replace = TRUE)
set.seed(321)
predicted <- sample(1:5, 1000, replace = TRUE)
goofcat(observed = observed, predicted = predicted)
\end{ExampleCode}
\end{Examples}
\inputencoding{utf8}
\HeaderA{homosoil\_globeDat}{Global Environmental Data}{homosoil.Rul.globeDat}
\keyword{datasets}{homosoil\_globeDat}
%
\begin{Description}
A data frame containing global-scale (0.5° resolution) climatic, lithological, and topographic variables.
\end{Description}
%
\begin{Usage}
\begin{verbatim}
data(homosoil_globeDat)
\end{verbatim}
\end{Usage}
%
\begin{Format}
\code{homosoil\_globeDat} is a \code{data.frame} with 62,254 rows and 58 columns.
\end{Format}
%
\begin{Details}
The dataset is structured on a global 0.5° resolution grid, integrating climate, topography, and lithology data.

\strong{Climate:} Sourced from ERA-40 reanalysis and CRU datasets, the climate variables include long-term monthly and seasonal averages of temperature, rainfall, solar radiation, and evapotranspiration. For each variable, 13 derived indicators were calculated: annual mean, wettest/driest month, annual range, wettest/driest quarter, hottest/coldest quarter, highest/lowest ET quarters, darkest/lightest quarters, and seasonality — yielding 52 total climatic predictors. More info: \url{http://www.ipcc-data.org/obs/get_30yr_means.html}

\strong{Topography:} Derived from the Hydro1k DEM (USGS), which includes elevation, slope, and the compound topographic index (CTI). Source: \url{https://lta.cr.usgs.gov/HYDRO1K}

\strong{Lithology:} Based on the global lithological map of Dürr et al. (2005), with 7 categorical values representing parent material groups:
\begin{enumerate}

\item{} Non-/semi-consolidated sediments
\item{} Mixed consolidated sediments
\item{} Siliciclastic sediments
\item{} Acid volcanic rocks
\item{} Basic volcanic rocks
\item{} Metamorphic and igneous complexes
\item{} Complex lithology

\end{enumerate}

\end{Details}
%
\begin{References}
\begin{itemize}

\item{} Dürr, H.H., Meybeck, M., and Dürr, S.H. (2005). \Rhref{http://dx.doi.org/10.1029/2005GB002515}{Lithologic composition of the Earth's continental surfaces derived from a new digital map emphasizing riverine material transfer}. \emph{Global Biogeochemical Cycles}, 19, GB4S10.

\end{itemize}

\end{References}
%
\begin{Examples}
\begin{ExampleCode}
library(ithir)
data(homosoil_globeDat)

# Structure of the dataset
str(homosoil_globeDat)

# Summary of the first few climatic variables
summary(homosoil_globeDat[, 1:6])
\end{ExampleCode}
\end{Examples}
\inputencoding{utf8}
\HeaderA{hunterCovariates}{Environmental Covariate Rasters for the Lower Hunter Valley, NSW}{hunterCovariates}
\aliasA{hunterCovariates\_A\_AACN}{hunterCovariates}{hunterCovariates.Rul.A.Rul.AACN}
\aliasA{hunterCovariates\_A\_elevation}{hunterCovariates}{hunterCovariates.Rul.A.Rul.elevation}
\aliasA{hunterCovariates\_A\_Hillshading}{hunterCovariates}{hunterCovariates.Rul.A.Rul.Hillshading}
\aliasA{hunterCovariates\_A\_light\_insolation}{hunterCovariates}{hunterCovariates.Rul.A.Rul.light.Rul.insolation}
\aliasA{hunterCovariates\_A\_MRVBF}{hunterCovariates}{hunterCovariates.Rul.A.Rul.MRVBF}
\aliasA{hunterCovariates\_A\_Slope}{hunterCovariates}{hunterCovariates.Rul.A.Rul.Slope}
\aliasA{hunterCovariates\_A\_TRI}{hunterCovariates}{hunterCovariates.Rul.A.Rul.TRI}
\aliasA{hunterCovariates\_A\_TWI}{hunterCovariates}{hunterCovariates.Rul.A.Rul.TWI}
\keyword{datasets}{hunterCovariates}
%
\begin{Description}
A set of GeoTIFF rasters representing environmental covariates across the full extent of the Lower Hunter Valley in New South Wales, Australia. These covariates are commonly used in digital soil mapping and terrain analysis.
\end{Description}
%
\begin{Format}
These raster layers (GeoTIFF format) are stored in the \code{inst/extdata/} directory of the package. Each raster has a spatial resolution of 25 m × 25 m and uses the WGS 84 UTM Zone 56 coordinate reference system.

Available files, all prefixed with \code{hunterCovariates\_A\_}, include:

\begin{description}

\item[\code{hunterCovariates\_A\_AACN.tif}] Elevation above channel network base level. Requires known stream network for interpretation.
\item[\code{hunterCovariates\_A\_elevation.tif}] Bare-earth digital elevation model (DEM), in meters above sea level.
\item[\code{hunterCovariates\_A\_Hillshading.tif}] DEM-derived hillshade raster with fixed illumination angle.
\item[\code{hunterCovariates\_A\_light\_insolation.tif}] Annual solar radiation potential derived from DEM (5-day timestep resolution).
\item[\code{hunterCovariates\_A\_MRVBF.tif}] Multi-resolution valley bottom flatness index for identifying depositional zones.
\item[\code{hunterCovariates\_A\_Slope.tif}] Slope angle in degrees (first derivative of elevation).
\item[\code{hunterCovariates\_A\_TRI.tif}] Topographic Ruggedness Index — measures terrain heterogeneity.
\item[\code{hunterCovariates\_A\_TWI.tif}] Topographic Wetness Index — indicates potential for water accumulation.

\end{description}

\end{Format}
%
\begin{Details}
These rasters cover the Hunter Wine Country Private Irrigation District (HWCPID), located in the Lower Hunter Valley (approx. 32.83°S, 151.35°E), \textasciitilde{}140 km north of Sydney. The area spans about 220 km², features a temperate humid climate, and is dominated by viticulture and dryland grazing systems. These layers are used in digital soil mapping studies and training material.
\end{Details}
%
\begin{Note}
This raster stack is used in the "Using R for Digital Soil Mapping" training course and is provided for demonstration and educational purposes.
\end{Note}
%
\begin{References}
\begin{itemize}

\item{} Gallant, J.C., Dowling, T.I. (2003). \Rhref{http://dx.doi.org/10.1029/2002WR001426}{A multiresolution index of valley bottom flatness for mapping depositional areas}. \emph{Water Resources Research}, 39(12), 1347.
\item{} Malone, B.P., Hughes, P., McBratney, A.B., Minasny, B. (2014). \Rhref{http://dx.doi.org/10.1016/j.geodrs.2014.08.001}{A model for the identification of terrons in the Lower Hunter Valley, Australia}. \emph{Geoderma Regional}, 1, 31–47.

\end{itemize}

\end{References}
%
\begin{Examples}
\begin{ExampleCode}
library(ithir)
library(terra)

# Load and plot the slope raster layer
slope <- rast(system.file("extdata/hunterCovariates_A_Slope.tif", package = "ithir"))
plot(slope, main = "Hunter Valley - Slope")
\end{ExampleCode}
\end{Examples}
\inputencoding{utf8}
\HeaderA{hunterCovariates\_sub}{Environmental Covariate Rasters for a Subset of the Lower Hunter Valley, NSW}{hunterCovariates.Rul.sub}
\aliasA{hunterCovariates\_sub\_AACN}{hunterCovariates\_sub}{hunterCovariates.Rul.sub.Rul.AACN}
\aliasA{hunterCovariates\_sub\_Elevation}{hunterCovariates\_sub}{hunterCovariates.Rul.sub.Rul.Elevation}
\aliasA{hunterCovariates\_sub\_Hillshading}{hunterCovariates\_sub}{hunterCovariates.Rul.sub.Rul.Hillshading}
\aliasA{hunterCovariates\_sub\_Landsat\_Band1}{hunterCovariates\_sub}{hunterCovariates.Rul.sub.Rul.Landsat.Rul.Band1}
\aliasA{hunterCovariates\_sub\_Light\_insolation}{hunterCovariates\_sub}{hunterCovariates.Rul.sub.Rul.Light.Rul.insolation}
\aliasA{hunterCovariates\_sub\_Mid\_Slope\_Position}{hunterCovariates\_sub}{hunterCovariates.Rul.sub.Rul.Mid.Rul.Slope.Rul.Position}
\aliasA{hunterCovariates\_sub\_MRVBF}{hunterCovariates\_sub}{hunterCovariates.Rul.sub.Rul.MRVBF}
\aliasA{hunterCovariates\_sub\_NDVI}{hunterCovariates\_sub}{hunterCovariates.Rul.sub.Rul.NDVI}
\aliasA{hunterCovariates\_sub\_Slope}{hunterCovariates\_sub}{hunterCovariates.Rul.sub.Rul.Slope}
\aliasA{hunterCovariates\_sub\_Terrain\_Ruggedness\_Index}{hunterCovariates\_sub}{hunterCovariates.Rul.sub.Rul.Terrain.Rul.Ruggedness.Rul.Index}
\aliasA{hunterCovariates\_sub\_TWI}{hunterCovariates\_sub}{hunterCovariates.Rul.sub.Rul.TWI}
\keyword{datasets}{hunterCovariates\_sub}
%
\begin{Description}
A suite of GeoTIFF rasters representing environmental covariates for a subset of the Lower Hunter Valley in New South Wales, Australia. These covariates are used in digital soil mapping and terrain analysis.
\end{Description}
%
\begin{Format}
The dataset consists of multiple raster layers (GeoTIFF format) located in the \code{inst/extdata/} directory of the package, each with a spatial resolution of 25 m × 25 m and a CRS of WGS 84 UTM Zone 56.

Available files are prefixed with \code{hunterCovariates\_sub\_} and include:

\begin{description}

\item[\code{hunterCovariates\_sub\_Terrain\_Ruggedness\_Index.tif}] Topographic ruggedness index (TRI).
\item[\code{hunterCovariates\_sub\_AACN.tif}] Elevation above channel network base level.
\item[\code{hunterCovariates\_sub\_Landsat\_Band1.tif}] Landsat 7 ETM+ Band 1 reflectance (0.45–0.52 µm).
\item[\code{hunterCovariates\_sub\_Elevation.tif}] Elevation in meters above sea level, derived from a DEM.
\item[\code{hunterCovariates\_sub\_Hillshading.tif}] Hillshade raster generated from the DEM using a fixed sun angle.
\item[\code{hunterCovariates\_sub\_Light\_insolation.tif}] Potential solar radiation calculated over a calendar year at 5-day intervals.
\item[\code{hunterCovariates\_sub\_Mid\_Slope\_Position.tif}] Slope position classification for crest/valley context.
\item[\code{hunterCovariates\_sub\_MRVBF.tif}] Multi-resolution valley bottom flatness index.
\item[\code{hunterCovariates\_sub\_NDVI.tif}] Normalized Difference Vegetation Index based on Landsat 7.
\item[\code{hunterCovariates\_sub\_TWI.tif}] Topographic Wetness Index (TWI).
\item[\code{hunterCovariates\_sub\_Slope.tif}] Slope angle in degrees.

\end{description}

\end{Format}
%
\begin{Details}
The subset area corresponds to the Hunter Wine Country Private Irrigation District (HWCPID) in the Lower Hunter Valley (approx. 32.83°S, 151.35°E), located \textasciitilde{}140 km north of Sydney. The HWCPID covers around 220 km² and supports viticulture and dryland grazing under a temperate, humid climate with \textasciitilde{}750 mm annual rainfall.

These covariates are used in spatial prediction tasks, terrain classification, and training examples in digital soil mapping.
\end{Details}
%
\begin{Note}
These rasters are used in the "Use R for Digital Soil Mapping" training material produced by the Soil Security Lab, The University of Sydney.
\end{Note}
%
\begin{References}
\begin{itemize}

\item{} Gallant, J.C., Dowling, T.I. (2003). \Rhref{http://dx.doi.org/10.1029/2002WR001426}{A multiresolution index of valley bottom flatness for mapping depositional areas}. \emph{Water Resources Research}, 39(12), 1347.
\item{} Malone, B.P., Hughes, P., McBratney, A.B., Minasny, B. (2014). \Rhref{http://dx.doi.org/10.1016/j.geodrs.2014.08.001}{A model for the identification of terrons in the Lower Hunter Valley, Australia}. \emph{Geoderma Regional}, 1, 31–47.
\item{} Soil Security Laboratory (2015). \emph{Use R for Digital Soil Mapping Manual}. The University of Sydney, Australia.

\end{itemize}

\end{References}
%
\begin{Examples}
\begin{ExampleCode}
library(ithir)
library(terra)

# Load and plot the elevation raster
elevation <- rast(system.file("extdata/hunterCovariates_sub_Elevation.tif", package = "ithir"))
plot(elevation, main = "Hunter Valley Subset - Elevation")
\end{ExampleCode}
\end{Examples}
\inputencoding{utf8}
\HeaderA{HV100}{Soil Point Data from the Hunter Valley, NSW, Australia}{HV100}
\keyword{datasets}{HV100}
%
\begin{Description}
A soil information \code{data.frame} with 100 observations from various locations in the Hunter Valley, NSW, Australia (approx. 32.83°S, 151.35°E). The data were collected using a stratified random sampling design in 2010, as described in Malone et al. (2011). Each row represents an observation at the 0–5 cm depth interval. Soil attributes include organic carbon, pH (1:5 soil:water), and electrical conductivity. Sites are labeled and georeferenced, with coordinates recorded in WGS84 / UTM Zone 56.
\end{Description}
%
\begin{Usage}
\begin{verbatim}
data(HV100)
\end{verbatim}
\end{Usage}
%
\begin{Format}
A \code{data.frame} with 100 rows and multiple columns:
\begin{description}

\item[\code{Site}] Site name or identifier
\item[\code{X}] Easting (UTM Zone 56)
\item[\code{Y}] Northing (UTM Zone 56)
\item[\code{SOC}] Soil organic carbon (\%)
\item[\code{pH}] Soil pH in water (1:5)
\item[\code{EC}] Electrical conductivity (dS/m)
\item[...] Other relevant soil or site variables

\end{description}

\end{Format}
%
\begin{Details}
This dataset represents a typical harmonized surface soil information table suitable for digital soil mapping and covariate analysis.
\end{Details}
%
\begin{References}
\begin{itemize}

\item{} Malone, B.P., de Gruijter, J.J., McBratney, A.B., Minasny, B., Brus, D.J. (2011). \Rhref{http://dx.doi.org/10.2136/sssaj2010.0280}{Using Additional Criteria for Measuring the Quality of Predictions and Their Uncertainties in a Digital Soil Mapping Framework}. Soil Science Society of America Journal, 75(3), 1032–1043.

\end{itemize}

\end{References}
%
\begin{Examples}
\begin{ExampleCode}
# Load and inspect the dataset
data(HV100)

# Basic structure and summary
str(HV100)
summary(HV100)
\end{ExampleCode}
\end{Examples}
\inputencoding{utf8}
\HeaderA{hvGrid25m}{Raster Grid of the Lower Hunter Valley, NSW, Australia}{hvGrid25m}
\keyword{datasets}{hvGrid25m}
%
\begin{Description}
A raster layer (GeoTIFF) representing pixel indices over the Lower Hunter Valley, NSW, Australia. The grid is aligned at 25 m × 25 m resolution and is commonly used as a spatial index for simulations and soil mapping exercises.
\end{Description}
%
\begin{Format}
\code{hvGrid25m} is a single-layer raster (GeoTIFF format) with:
\begin{itemize}

\item{} 860 rows
\item{} 676 columns
\item{} Pixel resolution of 25 m × 25 m

\end{itemize}

Each pixel contains a numeric index value. The coordinate reference system is \code{GDA94 / MGA Zone 56 (EPSG:28356)}.
\end{Format}
%
\begin{Details}
The grid covers the Hunter Wine Country Private Irrigation District (HWCPID) in the Lower Hunter Valley (32.83°S, 151.35°E), approximately 140 km north of Sydney. The region spans about 220 km² and features a temperate humid climate with \textasciitilde{}750 mm annual rainfall. Dominant land uses include viticulture and dryland grazing.

This grid is used in digital soil mapping exercises, including those taught through the "Using R for Digital Soil Mapping" course. It is useful for simulations, uncertainty mapping, and raster-based modeling workflows.
\end{Details}
%
\begin{References}
\begin{itemize}

\item{} Malone, B.P., Hughes, P., McBratney, A.B., Minasny, B. (2014). \Rhref{http://dx.doi.org/10.1016/j.geodrs.2014.08.001}{A model for the identification of terrons in the Lower Hunter Valley, Australia}. \emph{Geoderma Regional}, 1, 31–47.

\end{itemize}

\end{References}
%
\begin{Examples}
\begin{ExampleCode}
library(ithir)
library(terra)

# Load and plot the raster grid
hv.grid <- rast(system.file("extdata/hvGrid25m_grid.tif", package = "ithir"))
plot(hv.grid, main = "Hunter Valley 25m Grid Index")
\end{ExampleCode}
\end{Examples}
\inputencoding{utf8}
\HeaderA{hvPoints250}{Hunter Valley Soil Observation Points (n = 250)}{hvPoints250}
\keyword{datasets}{hvPoints250}
%
\begin{Description}
A \code{data.frame} containing coordinates for 250 randomly selected locations in the Lower Hunter Valley, NSW, Australia.
\end{Description}
%
\begin{Usage}
\begin{verbatim}
data(hvPoints250)
\end{verbatim}
\end{Usage}
%
\begin{Format}
A \code{data.frame} with 250 rows and 2 columns:
\begin{description}

\item[\code{x}] Easting (UTM Zone 56)
\item[\code{y}] Northing (UTM Zone 56)

\end{description}

\end{Format}
%
\begin{Details}
The area sampled is the Hunter Wine Country Private Irrigation District (HWCPID), located in the Lower Hunter Valley (approx. 32.83°S, 151.35°E), covering about 220 km\textasciicircum{}2. The HWCPID lies roughly 140 km north of Sydney in a temperate zone with warm, humid summers and relatively mild, humid winters. Average annual rainfall is approximately 750 mm. Land use includes extensive viticulture, followed by dryland grazing.
\end{Details}
%
\begin{References}
\begin{itemize}

\item{} Malone, B.P., Hughes, P., McBratney, A.B., Minasny, B. (2009). \Rhref{http://dx.doi.org/10.1016/j.geodrs.2014.08.001}{A model for the identification of terrons in the Lower Hunter Valley, Australia}. Geoderma Regional, 1, 31–47.

\end{itemize}

\end{References}
%
\begin{Examples}
\begin{ExampleCode}
data(hvPoints250)

# Summary
summary(hvPoints250)

# Simple plot
plot(hvPoints250, pch = 20, col = "darkgreen", main = "Hunter Valley Sample Points")
\end{ExampleCode}
\end{Examples}
\inputencoding{utf8}
\HeaderA{hvTerronDat}{Soil Point Data with Terron Class Labels from the Hunter Valley, NSW, Australia}{hvTerronDat}
\keyword{datasets}{hvTerronDat}
%
\begin{Description}
A \code{data.frame} of 1000 sites containing terron class information from various locations in the Hunter Valley, NSW, Australia (32.83°S, 151.35°E). Terrons are soil–landscape entities, conceptually similar to soil classes, but derived from a bottom-up clustering of relevant soil and environmental features. This approach is particularly useful for assessing viticultural suitability and delineating wine-growing subregions. The terron concept is discussed in Carré and McBratney (2005), and implemented in Malone et al. (2014).
\end{Description}
%
\begin{Usage}
\begin{verbatim}
data(hvTerronDat)
\end{verbatim}
\end{Usage}
%
\begin{Format}
\code{hvTerronDat} is a 1000-row \code{data.frame} with the following columns:
\begin{itemize}

\item{} \code{east}: Easting coordinate (GDA94 / MGA Zone 56)
\item{} \code{north}: Northing coordinate (GDA94 / MGA Zone 56)
\item{} \code{terron\_class}: A factor with 12 different Terron class labels

\end{itemize}

\end{Format}
%
\begin{Details}
The data are a random sample of Terron class assignments from the regional classification map published by Malone et al. (2014). Each entry represents a unique site with a corresponding terron label. These data are suitable for use in classification mapping, supervised learning, and regional planning.
\end{Details}
%
\begin{References}
\begin{itemize}

\item{} Carré, F., McBratney, A.B. (2005). \Rhref{http://dx.doi.org/10.1016/j.geoderma.2005.04.012}{Digital terron mapping}. \emph{Geoderma}, 128(3–4), 340–353.
\item{} Malone, B.P., Hughes, P., McBratney, A.B., Minasny, B. (2014). \Rhref{http://dx.doi.org/10.1016/j.geodrs.2014.08.001}{A model for the identification of terrons in the Lower Hunter Valley, Australia}. \emph{Geoderma Regional}, 1, 31–47.

\end{itemize}

\end{References}
%
\begin{Examples}
\begin{ExampleCode}
library(ithir)
data(hvTerronDat)
head(hvTerronDat)
table(hvTerronDat$terron_class)
\end{ExampleCode}
\end{Examples}
\inputencoding{utf8}
\HeaderA{HV\_dem}{Digital Elevation Model of the Hunter Valley, NSW}{HV.Rul.dem}
\keyword{datasets}{HV\_dem}
%
\begin{Description}
A \code{data.frame} containing spatial coordinates and elevation values representing a regular 100-metre resolution grid over the Lower Hunter Valley region of New South Wales, Australia. The coordinates are in WGS 84 / UTM Zone 56. When converted to a raster, the data form a digital elevation model (DEM).
\end{Description}
%
\begin{Usage}
\begin{verbatim}
data(HV_dem)
\end{verbatim}
\end{Usage}
%
\begin{Format}
A \code{data.frame} with three columns:
\begin{description}

\item[\code{x}] Easting (UTM Zone 56)
\item[\code{y}] Northing (UTM Zone 56)
\item[\code{elevation}] Elevation in metres above sea level

\end{description}

\end{Format}
%
\begin{Details}
This dataset can be readily converted to a raster using the \code{terra::rast()} function with \code{type = "xyz"}. It can support terrain analysis and serve as a covariate for digital soil mapping.
\end{Details}
%
\begin{References}
\begin{itemize}

\item{} Malone, B.P., Minasny, B., McBratney, A.B. (2017). \Rhref{https://link.springer.com/book/10.1007/978-3-319-44327-0}{Using R for Digital Soil Mapping}. Springer, Cham.

\end{itemize}

\end{References}
%
\begin{Examples}
\begin{ExampleCode}
library(terra)
data(HV_dem)

# Convert to raster and plot
dem_rast <- terra::rast(x = HV_dem, type = "xyz")
plot(dem_rast, main = "Hunter Valley DEM")
\end{ExampleCode}
\end{Examples}
\inputencoding{utf8}
\HeaderA{HV\_subsoilpH}{Hunter Valley Subsoil pH Data with Environmental Covariates}{HV.Rul.subsoilpH}
\keyword{datasets}{HV\_subsoilpH}
%
\begin{Description}
A \code{data.frame} containing 506 observations of soil pH from the Lower Hunter Valley, NSW, Australia. Each record represents the 60–100 cm depth interval and is associated with a set of intersected environmental covariates derived from digital elevation and Landsat data.
\end{Description}
%
\begin{Usage}
\begin{verbatim}
data(HV_subsoilpH)
\end{verbatim}
\end{Usage}
%
\begin{Format}
\code{HV\_subsoilpH} is a 506-row \code{data.frame}. The first two columns represent spatial coordinates in WGS 84 / UTM Zone 56. Soil pH for the 60–100 cm depth interval is stored in the next column, followed by several environmental covariates:

\begin{description}

\item[\code{Terrain\_Ruggedness\_Index}] Quantifies topographic heterogeneity. High values indicate rugged terrain.
\item[\code{AACN}] Elevation above channel network base level. Requires known stream network.
\item[\code{Landsat\_Band1}] Reflectance from Landsat 7 ETM+ Band 1 (0.45–0.52 µm).
\item[\code{Elevation}] Ground elevation (m), derived from a DEM.
\item[\code{Hillshading}] Hillshade based on DEM and fixed sun angle.
\item[\code{Light\_insolation}] Potential solar radiation, modeled over a year with 5-day intervals.
\item[\code{Mid\_Slope\_Position}] Relative classification of slope position (valley to crest).
\item[\code{MRVBF}] Multi-resolution Valley Bottom Flatness Index (Gallant \& Dowling, 2003).
\item[\code{NDVI}] Normalized Difference Vegetation Index from Landsat (B4-B3)/(B4+B3).
\item[\code{TWI}] Topographic Wetness Index.
\item[\code{Slope}] Slope angle in degrees (first derivative of elevation).

\end{description}

\end{Format}
%
\begin{Details}
The dataset covers the Hunter Wine Country Private Irrigation District (HWCPID), located in the Lower Hunter Valley (32.83°S, 151.35°E), about 140 km north of Sydney. The HWCPID spans \textasciitilde{}220 km², with a temperate humid climate (\textasciitilde{}750 mm annual rainfall). The region supports extensive viticulture and dryland grazing. This dataset is used in exercises for quantifying prediction uncertainty in digital soil mapping.
\end{Details}
%
\begin{Note}
This dataset is used in the "Using R for Digital Soil Mapping" course for exercises on uncertainty quantification.
\end{Note}
%
\begin{References}
\begin{itemize}

\item{} Gallant, J.C., Dowling, T.I. (2003). \Rhref{http://dx.doi.org/10.1029/2002WR001426}{A multiresolution index of valley bottom flatness for mapping depositional areas}. \emph{Water Resources Research}, 39(12), 1347.
\item{} Malone, B.P., Hughes, P., McBratney, A.B., Minasny, B. (2009). \Rhref{http://dx.doi.org/10.1016/j.geodrs.2014.08.001}{A model for the identification of terrons in the Lower Hunter Valley, Australia}. \emph{Geoderma Regional}, 1, 31–47.
\item{} Malone, B.P., Minasny, B., McBratney, A.B. (2017). \Rhref{https://link.springer.com/book/10.1007/978-3-319-44327-0}{Using R for Digital Soil Mapping}. Springer Cham, 262 pp.

\end{itemize}

\end{References}
%
\begin{Examples}
\begin{ExampleCode}
library(ithir)
data(HV_subsoilpH)
summary(HV_subsoilpH)
\end{ExampleCode}
\end{Examples}
\inputencoding{utf8}
\HeaderA{oneProfile}{Example Soil Profile for Soil Carbon Density}{oneProfile}
\keyword{datasets}{oneProfile}
%
\begin{Description}
A soil information \code{data.frame} representing a single soil profile for the variable soil carbon density. Each row corresponds to a horizon (depth interval).
\end{Description}
%
\begin{Usage}
\begin{verbatim}
data(oneProfile)
\end{verbatim}
\end{Usage}
%
\begin{Format}
A \code{data.frame} with 8 rows and 4 columns:
\begin{description}

\item[\code{id}] Profile identifier
\item[\code{top}] Upper depth (cm)
\item[\code{bottom}] Lower depth (cm)
\item[\code{carbon\_density}] Soil carbon density (e.g., kg/m\textasciicircum{}2)

\end{description}

\end{Format}
%
\begin{Details}
This dataset is useful for illustrating profile spline fitting, e.g., with \code{\LinkA{ea\_spline}{ea.Rul.spline}}, and is structured as a typical harmonized soil profile.
\end{Details}
%
\begin{References}
\begin{itemize}

\item{} Malone, B.P., Minasny, B., McBratney, A.B. (2017). \Rhref{https://link.springer.com/book/10.1007/978-3-319-44327-0}{Using R for Digital Soil Mapping}. Springer, Cham. 262 pp.

\end{itemize}

\end{References}
%
\begin{Examples}
\begin{ExampleCode}
data(oneProfile)

# Show profile structure
str(oneProfile)

# Fit spline (if using ithir)
# result <- ea_spline(oneProfile, var.name = "carbon_density")
# result$harmonised
\end{ExampleCode}
\end{Examples}
\inputencoding{utf8}
\HeaderA{plot\_ea\_spline}{Plot Soil Profile Outputs from \code{ea\_spline}}{plot.Rul.ea.Rul.spline}
\keyword{methods}{plot\_ea\_spline}
%
\begin{Description}
A plotting function for visualizing the fitted spline outputs from the \code{ea\_spline} function. It displays observed horizon data alongside continuous spline fits and/or spline-averaged values at harmonised depth intervals.
\end{Description}
%
\begin{Usage}
\begin{verbatim}
plot_ea_spline(splineOuts, d = t(c(0, 5, 15, 30, 60, 100, 200)), maxd, 
               type = 1, label = "", plot.which = 1)
\end{verbatim}
\end{Usage}
%
\begin{Arguments}
\begin{ldescription}
\item[\code{splineOuts}] A list; output from the \code{ea\_spline} function containing harmonised spline results and observed/predicted values.
\item[\code{d}] Numeric vector; harmonised standard depths used in the spline fitting.
\item[\code{maxd}] Numeric; maximum soil depth (cm) to be displayed on the plot.
\item[\code{type}] Integer (1, 2, or 3); controls which elements to display:
\begin{description}

\item[1] Observed values and the continuous spline (default).
\item[2] Observed values and spline averages at standard depths.
\item[3] Observed values, spline averages, and the continuous spline.

\end{description}


\item[\code{label}] Character string; optional label for the x-axis of the plot.
\item[\code{plot.which}] Integer; index of the profile to plot (when multiple profiles were processed by \code{ea\_spline}).
\end{ldescription}
\end{Arguments}
%
\begin{Value}
Generates a base R plot showing a single soil profile with user-specified spline elements overlaid. The function is designed to visualize one profile at a time.
\end{Value}
%
\begin{Note}
This is a companion function for \code{ea\_spline}. For plotting multiple profiles, this function should be called iteratively or wrapped in a custom loop or apply function.
\end{Note}
%
\begin{Author}
Brendan Malone
\end{Author}
%
\begin{Examples}
\begin{ExampleCode}
library(ithir)
library(aqp)

# Load example data and convert to SoilProfileCollection
data(oneProfile)
depths(oneProfile) <- Soil.ID ~ Upper.Boundary + Lower.Boundary

# Fit spline to the profile
eaFit <- ea_spline(oneProfile, var.name = "C.kg.m3.", 
                   d = t(c(0, 5, 15, 30, 60, 100, 200)), 
                   lam = 0.1, vlow = 0, show.progress = FALSE)

# Plot the fitted spline
plot_ea_spline(splineOuts = eaFit, 
               d = t(c(0, 5, 15, 30, 60, 100, 200)), 
               maxd = 200, 
               type = 1, 
               label = "Carbon Density")
\end{ExampleCode}
\end{Examples}
\inputencoding{utf8}
\HeaderA{plot\_soilProfile}{Plot Soil Profile Data}{plot.Rul.soilProfile}
\keyword{methods}{plot\_soilProfile}
%
\begin{Description}
A simple plotting function to visualize soil profile data from a \code{data.frame}. Each horizon is plotted as a bar, with its vertical extent representing the depth interval and its length representing the value of a soil property.
\end{Description}
%
\begin{Usage}
\begin{verbatim}
plot_soilProfile(data, vals, depths, label = "")
\end{verbatim}
\end{Usage}
%
\begin{Arguments}
\begin{ldescription}
\item[\code{data}] A \code{data.frame}; typically a soil profile table containing at least upper and lower depth columns, along with one or more soil property columns.
\item[\code{vals}] A numeric vector; the soil property values to be plotted, usually a column extracted from \code{data}.
\item[\code{depths}] A \code{data.frame} or \code{matrix} with two columns specifying the upper and lower boundaries of each horizon in the profile.
\item[\code{label}] Character string; optional label to display on the x-axis of the plot.
\end{ldescription}
\end{Arguments}
%
\begin{Value}
Returns a plot representing the soil profile, with bars for each horizon whose length corresponds to the supplied variable values, and height corresponds to the depth interval.
\end{Value}
%
\begin{Note}
This function is intended to plot one soil profile at a time. To visualize multiple profiles, it should be embedded in a loop or wrapper function.
\end{Note}
%
\begin{Author}
Brendan Malone
\end{Author}
%
\begin{Examples}
\begin{ExampleCode}
library(ithir)

# Load example profile data
data(oneProfile)

# Visualize the profile
plot_soilProfile(
  data = oneProfile,
  vals = oneProfile$C.kg.m3.,
  depths = oneProfile[, 2:3],
  label = names(oneProfile)[4]
)
\end{ExampleCode}
\end{Examples}
\inputencoding{utf8}
\HeaderA{precompute\_spline\_structures}{Precompute Spline Matrix Structures for Soil Profile Modeling}{precompute.Rul.spline.Rul.structures}
\keyword{methods}{precompute\_spline\_structures}
%
\begin{Description}
Builds the matrix structures needed to fit a mass-preserving spline model, based on a given set of soil depth intervals and a smoothing parameter.
These matrices are reused across all soil profiles or raster cells for efficient evaluation.
\end{Description}
%
\begin{Usage}
\begin{verbatim}
precompute_spline_structures(dIn, lam = 0.1)
\end{verbatim}
\end{Usage}
%
\begin{Arguments}
\begin{ldescription}
\item[\code{dIn}] numeric vector of input soil depth boundaries (e.g., \code{c(0,5,15,30,60,100,200)}).
\item[\code{lam}] numeric; spline smoothing parameter (\eqn{\lambda}{}). Smaller values produce smoother splines.
\end{ldescription}
\end{Arguments}
%
\begin{Value}
Returns a \code{list} containing precomputed matrix components used in mass-preserving spline fitting:
\begin{itemize}

\item{} \code{z} - coefficient matrix used to solve for spline coefficients
\item{} \code{rinv} - inverse of spline smoothness matrix
\item{} \code{q} - difference matrix for first derivatives
\item{} \code{u} - upper depths of each interval
\item{} \code{v} - lower depths of each interval
\item{} \code{delta} - thickness of each interval

\end{itemize}

\end{Value}
%
\begin{Author}
Brendan Malone
\end{Author}
%
\begin{SeeAlso}
\code{\LinkA{fit\_mpspline\_optimized}{fit.Rul.mpspline.Rul.optimized}}, \code{\LinkA{ea\_rasSp\_fast}{ea.Rul.rasSp.Rul.fast}}
\end{SeeAlso}
%
\begin{Examples}
\begin{ExampleCode}
# Standard SLGA input depths
dIn <- c(0,5,15,30,60,100,200)

# Compute the spline matrices
spline_info <- precompute_spline_structures(dIn = dIn, lam = 0.1)

# View structure
str(spline_info)
\end{ExampleCode}
\end{Examples}
\inputencoding{utf8}
\HeaderA{topo\_dem}{Example Digital Elevation Model as a Matrix}{topo.Rul.dem}
\keyword{datasets}{topo\_dem}
%
\begin{Description}
A small synthetic digital elevation model (DEM) represented as a numeric \code{matrix}. Row and column indices act as proxy spatial coordinates. Values represent elevation in metres.
\end{Description}
%
\begin{Usage}
\begin{verbatim}
data(topo_dem)
\end{verbatim}
\end{Usage}
%
\begin{Format}
A numeric \code{matrix} with 109 rows and 110 columns. Each value corresponds to a ground elevation.
\end{Format}
%
\begin{Details}
This dataset is used to exemplify procedures for generating random catenas or toposequences, as described in the book "Using R for Digital Soil Mapping".
\end{Details}
%
\begin{References}
\begin{itemize}

\item{} Malone, B.P., Minasny, B., McBratney, A.B. (2017). \Rhref{https://link.springer.com/book/10.1007/978-3-319-44327-0}{Using R for Digital Soil Mapping}. Springer, Cham.

\end{itemize}

\end{References}
%
\begin{Examples}
\begin{ExampleCode}
data(topo_dem)

# Basic inspection
str(topo_dem)
image(topo_dem, main = "Synthetic DEM", col = terrain.colors(20))
\end{ExampleCode}
\end{Examples}
\inputencoding{utf8}
\HeaderA{USYD\_dIndex}{Soil Drainage Index Observations and Predictions}{USYD.Rul.dIndex}
\keyword{datasets}{USYD\_dIndex}
%
\begin{Description}
A \code{data.frame} of 446 records, each representing a location with observed and predicted values for a soil drainage index. Locations are not georeferenced in this dataset.
\end{Description}
%
\begin{Usage}
\begin{verbatim}
data(USYD_dIndex)
\end{verbatim}
\end{Usage}
%
\begin{Format}
A \code{data.frame} with 446 rows and 2 columns:
\begin{description}

\item[\code{DI\_observed}] Observed soil drainage index (unitless or categorical)
\item[\code{DI\_predicted}] Predicted drainage index from a model

\end{description}

\end{Format}
%
\begin{Details}
This dataset is typical of model validation data in digital soil mapping, where observed values are compared to spatial model predictions. The dataset can be used to assess classification accuracy, correlation, or model bias.
\end{Details}
%
\begin{References}
\begin{itemize}

\item{} Malone, B.P., Minasny, B., McBratney, A.B. (2017). \Rhref{https://link.springer.com/book/10.1007/978-3-319-44327-0}{Using R for Digital Soil Mapping}. Springer, Cham.
\item{} Malone, B.P., McBratney, A.B., Minasny, B. (2018). \Rhref{https://doi.org/10.7717/peerj.4659}{Description and spatial inference of soil drainage using matrix soil colours in the Lower Hunter Valley, New South Wales, Australia}. PeerJ, 6, e4659.

\end{itemize}

\end{References}
%
\begin{Examples}
\begin{ExampleCode}
data(USYD_dIndex)

# View summary statistics
summary(USYD_dIndex)

# Plot observed vs predicted
with(USYD_dIndex, plot(DI_observed, DI_predicted,
     main = "Observed vs Predicted Drainage Index",
     xlab = "Observed", ylab = "Predicted"))
\end{ExampleCode}
\end{Examples}
\inputencoding{utf8}
\HeaderA{USYD\_soil1}{Random Selection of Soil Profile Data from New South Wales}{USYD.Rul.soil1}
\keyword{datasets}{USYD\_soil1}
%
\begin{Description}
A soil information \code{data.frame} containing 166 horizon observations across 29 soil profiles. Each row corresponds to a specific depth interval within a profile. Soil attributes such as organic carbon, pH, and electrical conductivity are included where available.
\end{Description}
%
\begin{Usage}
\begin{verbatim}
data(USYD_soil1)
\end{verbatim}
\end{Usage}
%
\begin{Format}
A \code{data.frame} with 166 rows and several columns:
\begin{description}

\item[\code{id}] Profile identifier
\item[\code{top}] Upper depth (cm)
\item[\code{bottom}] Lower depth (cm)
\item[\code{SOC}] Soil organic carbon (\%)
\item[\code{pH}] Soil pH in water (1:5)
\item[\code{EC}] Electrical conductivity (dS/m)
\item[...] Other soil or metadata fields

\end{description}

\end{Format}
%
\begin{Details}
This dataset is representative of a harmonized soil profile table commonly used for digital soil modelling, and is suitable for use with spline fitting tools such as \code{\LinkA{ea\_spline}{ea.Rul.spline}}.
\end{Details}
%
\begin{References}
\begin{itemize}

\item{} Malone, B.P., Minasny, B., McBratney, A.B. (2017). \Rhref{https://link.springer.com/book/10.1007/978-3-319-44327-0}{Using R for Digital Soil Mapping}. Springer, Cham. 262 pp.

\end{itemize}

\end{References}
%
\begin{Examples}
\begin{ExampleCode}
data(USYD_soil1)

# Overview
summary(USYD_soil1)

# Show profile IDs
table(USYD_soil1$id)
\end{ExampleCode}
\end{Examples}
\printindex{}
\end{document}
